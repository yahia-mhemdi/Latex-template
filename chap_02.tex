\chapter{Analyse et specification des besoins}

\section*{Introduction}

This chapter presents a detailed analysis of the functional and non-functional requirements of the LMS. It identifies system actors, summarises the main use cases, and describes the technical architecture and planning adopted during the internship.

\section{Functional requirements}

\subsection{Actors and roles}

The system defines the following primary actors:
\begin{itemize}[label=\textbullet,font=\normalsize]
  \item \textbf{Student}: registers, enrolls in courses, accesses lectures, asks questions, leaves reviews and manages a wishlist.
  \item \textbf{Teacher}: creates and updates courses, uploads lectures and media, views enrolled students, responds to Q\&A and manages coupons for their courses.
  \item \textbf{Admin / Platform operator}: moderates content, reviews payments and orders, manages users and site-wide settings through the admin dashboard.
\end{itemize}

\subsection{Key functional capabilities}

The LMS implements the following high-level features (examples taken from the application):
\begin{itemize}[label=\textbullet,font=\normalsize]
  \item User registration and JWT-based authentication (login/refresh/forgot-password flows).
  \item Course categories, course listing, course detail pages and search functionality.
  \item Curriculum management: sections and lectures with support for media uploads (video, files).
  \item Shopping cart, order creation and checkout with Stripe and PayPal integrations, coupon application and order history.
  \item Student-specific APIs: enrollments, student-specific course detail and progress, notes and reviews.
  \item Teacher-specific APIs: course creation/update, coupon management, notifications and earnings reports.
  \item File upload endpoint for media and video processing on the backend.
\end{itemize}

\subsection{Non-functional requirements}

\begin{itemize}[label=\textbullet,font=\normalsize]
  \item \textbf{Security}: JWT authentication, role-based access control and token blacklisting for secure session management.
  \item \textbf{Scalability}: Separation between frontend and API enables horizontal scaling of API servers and static hosting of the frontend.
  \item \textbf{Performance}: media uploads are handled asynchronously and large files are stored on the server file system (or object storage in production).
  \item \textbf{Maintainability}: clear modular structure for frontend components and backend Django apps facilitates incremental development and testing.
\end{itemize}

\section{Architecture overview}

The application follows a standard client-server architecture:
\begin{itemize}[label=\textbullet,font=\normalsize]
  \item \textbf{Frontend}: React (Vite), using libraries such as Axios for API requests, CKEditor for rich text fields, Bootstrap for layout and Zustand for lightweight state management.
  \item \textbf{Backend API}: Django 4.2 and Django REST Framework exposing JSON endpoints secured with Simple JWT. The backend includes app modules for authentication, core course models, user profiles and API logic.
  \item \textbf{Payments}: Stripe and PayPal integrations handled server-side; the backend exposes endpoints for checkout, payment success webhooks and coupon application.
  \item \textbf{Storage}: Development uses SQLite (db.sqlite3) and the local media folder for uploaded content; production deployments should move to a relational database (Postgres) and object storage for media.
\end{itemize}

\section{API highlights}

The backend exposes REST endpoints to manage authentication, courses and orders. Representative endpoints include:
\begin{itemize}[label=\textbullet,font=\normalsize]
  \item `user/token/`, `user/token/refresh/`, `user/register/` — authentication and user management.
  \item `course/course-list/`, `course/course-detail/<slug>/`, `course/search/` — course discovery.
  \item `course/cart/`, `order/create-order/`, `payment/stripe-checkout/<order_oid>/` — cart and payment flows.
  \item `student/course-detail/<user_id>/<enrollment_id>/` and `teacher/course-create/` — student and teacher specific resources.
\end{itemize}

\section{Planning and releases}

The project followed an iterative approach with three releases (each including two sprints) delivering the features described in later chapters.

