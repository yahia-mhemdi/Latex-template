\chapter{Release 1: User and Account Management}

\section*{Introduction}

This chapter documents the first release which focussed on user accounts, authentication and the initial course data model required to represent courses, categories and curriculum structure.

\section{Sprint 1: User registration and organisation setup}

\subsection{Sprint objectives}
The goal of the first sprint was to provide secure user registration and the basic data model for courses and categories so that instructors can be created and minimal course listings can be displayed.

\subsection{Backlog (examples)}
\begin{longtable}{|m{1cm}|m{8cm}|m{2cm}|m{2cm}|}
\hline
\textbf{Id} & \textbf{Feature} & \textbf{Priority} & \textbf{Estimate (days)} \\
\hline
1 & Implement user registration endpoint and JWT login & 1 & 3 \\
\hline
2 & Create course and category models with API endpoints & 1 & 3 \\
\hline
3 & Frontend pages for sign up and login (React + form validation) & 2 & 3 \\
\hline
4 & Profile endpoint and frontend profile editing & 3 & 2 \\
\hline
\end{longtable}

\subsection{Implementation notes}
Backend implementation uses Django apps to isolate authentication (`userauths`) and course resources (`api` and `core`). Authentication is implemented with `rest_framework_simplejwt` and supporting endpoints for token refresh and password reset. The frontend uses `react-router` and protected routes to store and forward JWT tokens received after login.

\section{Sprint 2: Roles and user management}

\subsection{Sprint objectives}
Implement role-based access control (student, teacher, admin) and administrative interfaces for user management. Provide initial admin customisations (jazzmin) for site operators.

\subsection{Backlog (examples)}
\begin{longtable}{|m{1cm}|m{8cm}|m{2cm}|m{2cm}|}
\hline
1 & Implement role assignment and permission checks in API views & 1 & 2 \\
\hline
2 & Admin interface customisation (jazzmin) & 2 & 2 \\
\hline
3 & Frontend access control and role-aware navigation & 2 & 2 \\
\hline
\end{longtable}

\section*{Conclusion}
Release 1 established the authentication and user management foundations and the minimal course model needed for subsequent releases.

