\chapter*{Conclusion générale}
\addcontentsline{toc}{chapter}{Conclusion générale}
\markboth{\textbf{Conclusion générale}}{}




Ce rapport présente notre projet de fin d’étude au sein de Tek-Up University et marque l’aboutissement de notre parcours d’enseignement supérieur.
Il nous a permis de mettre en pratique les connaissances acquises tout au long de notre cursus universitaire, y compris pendant notre stage au sein de l’entreprise Tek-Up.
L’entreprise Tek-Up nous a proposé de développer une application de Learning Management System, ayant pour objectif de prendre en charge la Learning Management System en simplifiant la gestion du personnel et certaines tâches administratives.

Dans ce rapport, nous avons détaillé l’ensemble des étapes de réalisation du projet, pendant lequel nous avons pris soin de construire notre application de façon incrémentale et itérative en utilisant la méthode agile Scrum. Cette méthode, largement utilisée de nos jours notamment par les entreprises numériques innovantes, permet d’obtenir d’améliorer la célérité et la qualité du processus de développement d’un projet et de répondre au mieux à la satisfaction des clients. 

Durant le stage réalisé auprès de Tek-Up, nous avons pu réaliser six modules de l’application LMS portant respectivement sur :

    \begin{itemize}[label=\textbullet,font=\normalsize]

    \item La gestion de l'entreprise
    \item La gestion des utilisateurs et authentification
    \item La gestion des congés et des autorisations de sortie
    \item La gestion de présence et compte rendu d'activité
    \item La gestion des achats et dépenses
    \item Le Dashboard
    
    \end{itemize}

L’ensemble de ces fonctionnalités se trouvant ainsi pris en charge par l’application, la Learning Management System devient plus simple et plus fluide au quotidien pour les utilisateurs de l’application en question. 

Ce projet n’est pas voué à rester figé dans le temps et des pistes d’amélioration nous apparaissent déjà sous la forme de nouvelles fonctionnalités qui pourront être ajoutées afin de compléter l’excellente base que constitue déjà notre application, il s’agira par exemple de la gestion des salaires et des factures, d’une implémentation de la partie planification ayant pour objectif l’organisation des tâches des employées ainsi que, l’élaboration d’une fonctionnalité de messagerie instantanée qui permettra d’améliorer la communication entre les membres de l’entreprise.
    


    

















%À travers ce rapport, nous avons présenté le travail que nous avons effectué lors de notre stage chez Tek-Up en Tunisie. Nous avons créé une application qui suit et répond aux besoins des utilisateurs tout en suivant nos spécifications.
%D'un point de vue technique, ce stage est très intéressant. Nous avons appris les méthodes de travail et les outils qui nous aident à réaliser le projet.
%Sous la supervision de notre encadrant technique de notre société, nous avons réalisé une formation sur le langage de développement Angular. Cette formation est très intéressante. Elle nous a permis d'obtenir un nouveau langage de développement, qui nous aidera dans nos futures carrières.
%D’un point de vue conception, ce stage nous a donné un grande plus en nous concentrant sur le UX design (Expérience utilisateur) pour créer une application facile à utiliser et ergonomique encadrée par notre encadrant design.
%Enfin, notre travail n'est qu'un début, le projet ne s'arrêtera pas à ce niveau, et certaines fonctions seront ajoutées, comme la partie salaire et la partie planification pour organiser les tâches entre les employées.

