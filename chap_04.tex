\chapter{Release 2: Course Consumption and Interaction}

\section*{Introduction}

Release 2 implemented the features that allow students to discover, consume and interact with course content: curriculum and lecture pages, progress tracking, Q\&A and review systems.

\section{Sprint 3: Curriculum and lecture delivery}

\subsection{Sprint objectives}
The sprint focused on implementing a curriculum model (sections and lectures), lecture upload and streaming support, and the student-facing course detail and progress endpoints.

\subsection{Backlog (examples)}
\begin{longtable}{|m{1cm}|m{8cm}|m{2cm}|m{2cm}|}
\hline
1 & Create curriculum and lecture models, APIs and upload endpoints & 1 & 4 \\
\hline
2 & Frontend lecture player and progress tracking & 1 & 4 \\
\hline
3 & Video upload handling and server-side processing (thumbnails) & 2 & 3 \\
\hline
\end{longtable}

\section{Sprint 4: Interaction features (Q\&A, reviews, wishlist)}

\subsection{Sprint objectives}
Add features that enable student–teacher interaction and quality signals: a lightweight Q\&A or discussion flow per course, course reviews and ratings, and a wishlist for students to save courses.

\subsection{Backlog (examples)}
\begin{longtable}{|m{1cm}|m{8cm}|m{2cm}|m{2cm}|}
\hline
1 & Implement student review API and frontend review forms & 1 & 2 \\
\hline
2 & Implement question-answer endpoints and discussion UI & 1 & 3 \\
\hline
3 & Wishlist endpoints and frontend integration & 2 & 2 \\
\hline
\end{longtable}

\section*{Conclusion}
Release 2 delivered the essential course consumption and interaction features necessary for a functioning LMS platform.

