\chapter*{Introduction generale}
\addcontentsline{toc}{chapter}{Introduction generale} % to include the introduction to the table of content
\markboth{Introduction generale}{} %To redefine the section page head

This report documents the design and implementation of a Learning Management System (LMS) developed during the internship period. The LMS provides a web-based platform for creating, publishing and selling online courses, managing users (instructors and students), handling enrollments, and processing payments.

The implementation is divided into a React-based frontend (built with Vite) and a Django REST Framework backend. The application implements authentication using JWT, course and curriculum management, a shopping cart and checkout flow with Stripe and PayPal integrations, content upload (including video support), discussion threads, reviews, wishlist, and an administrative dashboard used by platform operators.

The internship was supervised by Aicha ben Jrad and carried out as part of the curriculum of Tek-Up University. The main objectives were:
\begin{itemize}[label=\textbullet,font=\normalsize]
  \item Implement a full-stack LMS with clear separation between frontend and API backend.
  \item Provide secure user authentication (students and teachers) and role-based features.
  \item Support course creation, curriculum management, video uploads and course consumption.
  \item Implement a shopping and payment flow with coupons and order tracking.
  \item Deliver an admin interface for content moderation and operational tasks.
\end{itemize}

The report is organised as follows:
\begin{itemize}[label=\textbullet,font=\normalsize]
  \item Chapter 1: Project context and motivation, study of similar systems, and the overall project methodology.
  \item Chapter 2: Requirements analysis and specification (actors, functional and non-functional requirements, system architecture and planning).
  \item Chapter 3: Release 1 — user and account management, authentication and basic course model implementation.
  \item Chapter 4: Release 2 — course consumption features: curriculum, lectures, student activity and interaction (Q&A, reviews), and course purchasing flows.
  \item Chapter 5: Release 3 — payments, orders, teacher features (course creation and management), and the administrative dashboard.
\end{itemize}

