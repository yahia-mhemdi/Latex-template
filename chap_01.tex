\chapter{Cadre du projet}


% Une section

% Exemple d'une section qui porte une référence à une bibliographie
% NB: il faut bien suivre le syntaxe pour ne pas tomber dans le cas où il y a une référence dans la table des matières.

\section*{Introduction}

Dans ce chapitre, nous situons le projet dans son cadre global. Nous commençons par présenter l'entreprise dans laquelle nous avons réalisé notre stage de projet de fin d’étude. Puis, nous expliquons les objectifs du projet et nous analysons trois applications web existantes. Enfin, nous présentons la méthodologie de projet que nous adoptons ainsi que l'environnement de travail.

\section{Cadre général du projet} 
    \subsection{Présentation de l’organisme d’accueil}
        \subsubsection{Tek-Up Group}
    
            L’entreprise Tek-Up Group fondée en 2014 est au carrefour de la technologie des
            systèmes d'information, du conseil et de la finance de marché. Grâce à une stratégie claire, elle
            propose une assistance globale et cohérente aussi bien en conseil qu’en formation et expertise.\\
            
            Le groupe est basé à Tunis et dispose de bureaux à Paris et kowloon 
            
            
            \begin{figure}[h]
            \centering
            \\includegraphics[scale=0.7]{}\\[0.5cm]
             \caption{Carte filiale de Tek-Up }
            \end{figure}\

        \subsubsection{Service}
             L’entreprise Tek-Up Group fournit une expertise et des techniques fonctionnelles,
            guidée en permanence par l’envie de résoudre les problèmes des entreprises de ses clients en
            fournissant des solutions et des conseils qui peuvent être classifiés selon trois axes : [1]
             \begin{itemize}
                \item  [$\bullet$] \textbf{Solutions publicitaires :} Filière qui..
                \item  [$\bullet$] 	\textbf{Consulting :} A travers... 
                \item  [$\bullet$] 	\textbf{Produits et services pour la Finance et l'IT :} De la direction générale..
            \end{itemize}

    
    \subsection{Contexte et Objectifs du projet}
        \subsubsection{Contexte}
        La Learning Management System (HRM) occupe une place importante dans l'entreprise car elle améliore la communication entre les différents intervenants et favorise une parfaite compréhension de l'entreprise, favorisant ainsi le développement des organisations professionnelles. Dans ce contexte l'idée de notre projet fin d'étude est de concevoir et mettre en place une application de Learning Management System LMS
        
        \subsubsection{Objectifs du projets}
        
        L'enjeu principal de notre application est la Learning Management System est d'assurer toutes les tâches qui permettent d'accompagner et d'aider les différents services internes en matière de Learning Management System et non seulement ça elle permet aussi à faciliter le travail du département finance grâce au module achat et dépense. 
 
        Il s'agit aussi d'un module de gestion des achats et dépenses, ainsi qu'une partie administrative. 
 
        Le soutien des collaborateurs dans leurs conditions de travail dans leurs vies dans l'entreprise est un objectif essentiel et d'offrir au manager une vue globale.

\section{Étude de l'existant} 
L'étude de l'existant est une étape indispensable, elle permet d'extraire les forces et les faiblesses des applications existantes. Cela nous aidera à la réalisation de notre projet. Nous avons choisi d'analyser trois applications existantes.

\begin{itemize}[label=\textbullet,font=\normalsize]
\item L’application « Open Time Clock »
\item L’application « IceHRM »
\item L’application « SnapHRM »
\end{itemize}

\subsection{Étude de l'application Open Time Clock}
Adresse (URL) : https://www.opentimeclock.com/ 

\subsubsection{Description}
Cette application a pour but le suivi des pointages, des présences, des congés payés ainsi que la gestion des comptes des employés, la gestion des factures, la gestion des événements... 
Le public ciblé de cette application est les entreprises (petite, moyenne ou grande entreprise).

Il existe une application pour le Web et une version pour le mobile (Android/iOS).  


\begin{itemize}[label=\textbullet]
        \item \textbf{remarque:}
La société ‘Tek-Up’ utilise l’application ‘Open time Clock’ comme outil de Learning Management System. La question qui se pose pourquoi Tek-Up veut changer cette application ?
\begin{itemize}[label=\textbullet,font=\normalsize]

\item L’application "Open time Clock" n’est pas bien sécurisée puisqu'il existe des sessions avec login seulement et sans mot de passe.
\item L'entreprise veut développer une application avec accès gratuit pour les startups et les mini entreprises. 
\item Cette application manque des fonctionnalités comme la gestion des achats et des dépenses, Dashboard manager…
\item L'entreprise Tek-Up veut avoir un environnement intérieur qui s'adapte à leurs besoins.

\end{itemize}
\end{itemize}


\begin{itemize}[label=\textbullet]
\item \textbf{Interface du site:} C'est l’interface d’accueil qui contient la description du site. 

\begin{figure}[H]
\centering
\frame{\\includegraphics[width=0.7\columnwidth]{}}
\caption{Interface du site Open Time Clock}
\label{fig:OpenTime}
\end{figure}

\end{itemize}

\subsubsection{Étude fonctionnelle}

Les principales fonctionnalités proposées par l'application, tout en les reliant aux acteurs qui en bénéficient.


\begin{itemize}[label=\textbullet]
        \item \textbf{Employé:}
      
 \begin{itemize}
 \item Pointer (clock in/clock out) possibilité de pointer en utilisant un nom, la reconnaissance faciale, un pin ou un QR code.
 \item Consulter ses heures de travail.
 \item Consulter sa fiche de paie (les heures payées).
\item Échanger des messages avec toute l’équipe du travail.
\item Modifier son profil.
\item Enregistrer sa localisation.
\item Consulter les quarts de travail programmés.
\item Demander une autorisation de sortie/un congé.
\end{itemize}

                   
\item \textbf{Administrateur:}                   
  
\begin{itemize}
\item Gérer ses employés ainsi que les managers (ajouter, supprimer et modifier des profils).
\item Consulter les absences et les retards des employés et des managers.
\item Consulter les fiches de paie.
\item Gérer les quarts de travail programmés.
\item Gérer les départements (ajouter, supprimer et modifier).
\item Gérer les autorisations de sorties/congés.
\item Consulter les absences.
\item Consulter la localisation lors du pointage.
\end{itemize}

\end{itemize}


\subsubsection{Points forts et points faibles}

      
    \begin{longtable}{|m{8cm}|m{8cm}|}
                \hline 
                     \textbf{Points forts}
                    & \textbf{Points faibles}\\
                \hline
                \endhead
                 \endfoot
                  \endlastfoot
                   \hline  
                
         
                     \begin{itemize}
                        \item Facile à utiliser.
                        \item Facile à installer.
                        \item Application gratuite/payante.
                        \item Un temps de chargement moyen des pages (Les liens s’ouvrent rapidement).
                        \item Les employés peuvent utiliser un ordinateur, tablette/iPad et smartphone (Android and iPhone) pour le pointage.
                        
                    \end{itemize}
                    &\begin{itemize}
                        \item Le site n’est pas sécurisé.
                        \item La composition des différentes interfaces du site n’est pas homogène.
                        \item L'application dispose deux acteurs seulement
                    \end{itemize}\\
                \hline 

                   
                \captionsetup{justification=centering,margin=2cm}
                \caption{Points forts et points faibles de l'application Open Time Clock}
            
            \end{longtable}

\subsection{Étude de l’application IceHRM}
Adresse (URL) : http://icehrm.com

\subsubsection{Description}
‘ICE HRM' est un système de Learning Management System qui souhaite simplifier la Learning Management System. 
Il existe une version Web et une version mobile (Android/iOS).


\begin{itemize}[label=\textbullet]
        \item \textbf{Interface du site:}
La figure \ref{fig:iceHRM} représente l’interface d’accueil du site vitrine IceHRM.

\begin{figure}[H]
\centering
\frame{\\includegraphics[width=0.7\columnwidth]{}}
\caption{Interface du site iceHRM}
\label{fig:iceHRM}
\end{figure}


\end{itemize}

\subsubsection{Étude fonctionnelle}
Les principales fonctionnalités offertes par l'application, tout en les associant aux acteurs qui en bénéficient.

\begin{itemize}[label=\textbullet]
\item \textbf{Manager:}
               

\begin{itemize}
\item Gérer les employés (ajouter, supprimer et modifier les profils des employés).
\item Gérer les fiches de paie.
\item Gérer et consulter les congés (approuver ou ignorer une demande de congé).
\item Gérer les candidatures.
\item Gérer les formations.
\item Gérer les clients (enregistrer les informations des clients pour chaque projet).
\item Gérer les départements (ajouter ou supprimer des départements).
\item Consulter les absences.
\item Suivre les candidatures.
\item Suivre  l’avancement des projets.
\item Gérer les dépenses (approuver ou ignorer les dépenses enregistrées par les employés).
\end{itemize}
                 
\item \textbf{Employé:}  
 
\begin{itemize}
\item Consulter ses absences.
\item Demander une autorisation de sortie/un congé.
\item Consulter les événements.
\item Demander une autorisation pour participer à un événement.
\item Échanger des messages (avec toute l’équipe).
\item Modifier son profil.
\item Suivre les projets sur lesquels il travaille.
\item Enregistrer ses dépenses.
\end{itemize}

\end{itemize}
                    
            

\subsubsection{Points forts et points faibles}
 \begin{longtable}{|m{8cm}|m{8cm}|}
            
                \hline 
                   \textbf{Points forts} & \textbf{Points faibles}\\
                \hline
                \endhead
                 \endfoot
                  \endlastfoot
                \hline 
                
                \begin{itemize}
                        \item Les liens s’ouvrent rapidement.
                        \item L’employé ou l’administrateur peuvent naviguer facilement.
                    \end{itemize}
                    &\begin{itemize}
                        \item Application payante.
                        \item La page d’accueil est trop compliquée.
                        \item L'application dispose deux acteurs seulement

                    \end{itemize}\\
                \hline 
                
                 \captionsetup{justification=centering,margin=2cm}
                 \caption{Points forts et points faibles de l'application IceHRM}
            \end{longtable}
            
\subsection{Étude de l’application snapHRM}

\subsubsection {Description}
'Snaphrm' est une application de Learning Management System pour les petites et moyennes entreprises, avec lesquelles vous pouvez gérer les congés, les présences, la participation aux événements, la masse salariale, les dépenses, les projets ...

Il existe une version Web et une version mobile (Android/iOS).


\begin{itemize}[label=\textbullet]

\item \textbf{Interface de l'application:}
La figure \ref{fig:snapHRMapp} représente l’interface d’accueil de l’application qui contient un dashboard qui présente toutes les fonctionnalités pour l’admin.

\begin{figure}[H]
\centering
\frame{\\includegraphics[width=0.7\columnwidth]{}}
\caption{Interface de l'application snapHRM}
\label{fig:snapHRMapp}
\end{figure}

\end{itemize}


\subsubsection{Étude fonctionnelle}
Les principales fonctionnalités proposées par l'application, tout en les reliant aux acteurs qui en bénéficient.

\begin{itemize}[label=\textbullet]
\item \textbf{Manager:}
                   
\begin{itemize}
\item Gérer son profil.
\item Gérer les employés (ajouter, supprimer et modifier les profils des employés).
\item Gérer les évènements et les vacances.
\item Gérer les fiches de paie.
\item Gérer et consulter les congés (approuver ou ignorer une demande de congé).
\item Gérer les départements (ajouter ou supprimer des départements).
\item Consulter les absences.
\item Suivre l’avancement des tâches de chaque employé.
\item Gérer les dépenses (approuver ou ignorer les dépenses enregistrées par les employés).
\item Gérer les candidatures.
\end{itemize}
                 
                 
\item \textbf{Employé:} 

\begin{itemize}
\item Gérer son profil (consulter et modifier ses informations).
\item Pointer (en cliquant).
\item Consulter ses absences.
\item Demander une autorisation de sortie/un congé.
\item Consulter les évènements (calendrier).
\item Demander une autorisation pour participer à un événement.
\item Échanger des messages (avec toute l’équipe).
\item Suivre les projets sur lesquels il travaille.
\item Enregistrer ses dépenses.
\item Consulter les membres de chaque département.
                        
\end{itemize}
\end{itemize}
             
\subsubsection{Points forts et points faibles}
 \begin{longtable}{|m{8cm}|m{8cm}|}
                
            
                \hline 
                    \textbf{Points forts} & \textbf{Points faibles}\\
                \hline
                \endhead
                 \endfoot
                  \endlastfoot
                \hline  
                
                    \begin{itemize}
                        \item L’employé ou l’administrateur peuvent naviguer facilement.
                        \item Application simple et facile à utiliser.
                    \end{itemize}
                    &\begin{itemize}
                        \item Application payante (il y a une version gratuite si le nombre d'employés ne dépasse pas 5
                        \item L'application dispose deux acteurs seulement
                    \end{itemize}\\
                \hline 
                \captionsetup{justification=centering,margin=2cm}
                \caption{Points forts et points faibles de l'application SnapHRM}
            \end{longtable}
            


\section{Synthèse pour le choix de la solution}
Après l'analyse de trois applications, nous avons conclu que l’application de la Learning Management System doit répondre aux besoins suivants :

\begin{itemize}[label=\textbullet,font=\normalsize]
        \item Utiliser le maximum possible d'interface simple et bien organiser pour faciliter l’expérience de l’utilisateur.
        \item Assurer que toutes les fonctionnalités seront gratuites.
        \item Répartir l'application selon le système des rôles.
        \item Faciliter le travail d'un responsable des ressources humaines et aider l'administrateur (PDG de l'entreprise) à consulter la situation de son entreprise.
        \item Réaliser une application spécifique ciblant la Learning Management System d'une mini entreprise ou d'une startup répondant à certains besoins fonctionnels.
        
      \end{itemize}



\section{Méthodologie de gestion de projet à adopter}

\subsection{Méthodologie de SCRUM}
Nous recherchons toujours la méthode la plus efficace et la plus rapide de mettre en œuvre notre projet. Pour cela, nous utilisons des méthodes agiles. Scrum est le cadre agile le plus simple. Scrum garantit la meilleure vue d'ensemble de notre projet et vise à réduire les difficultés, telles que le manque de planification, le travail est effectué à travers un cycle court appelé Sprint. Dans Sprint, notre équipe travaille à partir d'une liste d'éléments appelée Backlog" \textbf{[B1]}.

Nous avons choisit la méthodologie \textbf{SCRUM} qui fait partie de la méthodologie \textbf{Agile} pour différentes raisons parmi lesquelles :

\begin{itemize}[label=\textbullet]
        \item La transparence.
        \item La tolérance aux différents changements.
        \item Les besoins ne sont pas bien connus au départ ils ont été changés par la suite. 
        \item La présence du mêlée quotidien qui permet de bien résoudre les problèmes et de trouver des solutions rapidement. 
     
\end{itemize}

La figure \ref{fig:Scrum-process} représente le parcours de la méthodologie de SCRUM.

\begin{figure}[H]
\centering
\frame{\\includegraphics[width=1\columnwidth]{}}
\caption{Cycle de vie de la méthodologie SCRUM \cite{webArticle2}}
\label{fig:Scrum-process}
\end{figure}

\subsection{Les rôles dans SCRUM}
La méthodologie d'agile SCRUM implique trois rôles principaux qui sont:
\begin{itemize}[label=\textbullet]
        \item \textbf{Product owner:} C’est le représentant des clients et des utilisateurs et c'est lui qui est l'expert métier de l'équipe. 
        C’est à lui de définir et prioriser la liste des fonctionnalités du produit et effectuer l’analyse nécessaire pour la  prise des décisions.
        
        
        \item \textbf{Scrum master:}
        C'est le garant de la méthodologie de SCRUM, qui garantit que tout le monde peut maximiser ses capacités en éliminant les obstacles, et en protégeant l'équipe des perturbations externes. Par ailleurs il garantit que l'équipe chargée du projet adopte les principes et les valeurs de SCRUM. 

        
        \item \textbf{Équipe:} L'équipe rassemble tous les rôles généralement nécessaires à un projet, elle est organisée et reste inchangée pendant la durée d'un sprint. 

\end{itemize}



\subsection{Méthodologie de conception à adopter}

La modélisation du système d'information a pour but de permettre aux entreprises de communiquer avec des services ou des entreprises spécialisées dans l'informatique et de décrire leurs opérations et leurs besoins. Le modèle peut être résumé de manière claire et compréhensible pour tout le monde.
Pour cela, nous avons choisi le langage de modélisation UML (Unified Modeling Language), car il s'appuie sur la standardisation et pour la diversité de ses diagrammes. Ce qui permet de réaliser une analyse détaillée des besoins, des vues statiques et dynamiques...





\section*{Conclusion}
Dans ce chapitre, nous avons présenté le projet dans son cadre général, nous passons maintenant à la spécification des besoins.







