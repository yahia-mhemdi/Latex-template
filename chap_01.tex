\chapter{Cadre du projet}

\section*{Introduction}

This chapter places the LMS project within its general context. It presents the hosting organization, the objectives of the internship project and the motivation for delivering a Learning Management System that supports course creation, student enrollment and payments.

\section{General context and project objectives}

\subsection{Host organisation}

The internship was carried out with Tek-Up University in collaboration with the project supervisors. Tek-Up provided access to development infrastructure, technical review and mentoring throughout the internship.

\subsection{Project motivation}

Online education platforms have become a central channel for delivering both formal and informal training. The LMS developed in this project aims to provide a lightweight, extensible platform that allows instructors to create courses composed of sections and lectures, students to enroll and follow content, and platform operators to manage payments and content lifecycle.

\subsection{Project objectives}

The principal objectives of the project were:
\begin{itemize}[label=\textbullet,font=\normalsize]
  \item Build a modern, responsive web front-end using React and Vite.
  \item Implement a RESTful API with Django REST Framework to expose course, user and order resources.
  \item Secure user authentication using JWT and role-based access control (student / teacher / admin).
  \item Provide shopping cart, coupon and payment integrations (Stripe and PayPal) for paid courses.
  \item Deliver an administrative dashboard for content moderation and operational tasks.
\end{itemize}

\section{Study of existing solutions}

To position the LMS, three representative solutions were considered:
\begin{itemize}[label=\textbullet,font=\normalsize]
  \item \textbf{Moodle}: an open-source LMS widely used in education; strong in course management and extensibility but heavyweight for small teams.
  \item \textbf{Open edX}: large-scale open-source platform focused on MOOCs and institutional deployments; powerful but complex to operate.
  \item \textbf{Udemy (platform model)}: an example of a commercial course marketplace with instructor-managed content and payment flows.
\end{itemize}

The implemented LMS takes a middle path: it aims for a simpler operational footprint than enterprise offerings while including essential marketplace features (courses, checkout, coupons, teacher accounts).

\section{Conclusion}

This chapter explained the project context and defined clear objectives for the LMS development. The next chapter specifies detailed functional and non-functional requirements.

